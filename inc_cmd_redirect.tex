\begin{table}[htbp]
\begin{tabular}[width=\linewidth]{|c|p{8cm}|}
\hline
Saisie & Signification\\
\hline
\verb'commande' & La commande est ex�cut�e normalement, le symbole
                  d'invite sera affich� lorsqu'elle se terminera.\\
\hline

\verb'commande &' & La commande est ex�cut�e en arri�re-plan, ce qui
                    signifie qu'elle n'a en principe pas acc�s au terminal. 

                    Le shell reprend donc la main imm�diatement, et affiche
                    son symbole d'invite alors que la commande continue � 
                    s'ex�cuter en t�che de fond.

                    Bash affiche une ligne du type
                    \verb'[1] 2496'

                    indiquant le num�ro du job � l'arri�re-plan suivi du PID
                    c'est-�-dire de l'identifiant de processus.
                    Lorsque la commande se termine, Bash affichera une ligne :

                    \verb'[1]+    Done commande'

                    juste avant de proposer un nouveau symbole d'invite.\\
\hline

\verb'commande > fichier' & La commande est ex�cut�e, mais sa sortie standard
                            est dirig�e vers le fichier indiqu�. Seule la
                            sortie d'erreur s'affiche � l'�cran.\\
\hline

\verb'commande >> fichier' & La sortie standard est ajout�e en fin de fichier
                             sans en �craser le contenu.\\
\hline

\verb'commande < fichier' & La commande est ex�cut�e, mais ses informations 
                            d'entr�e seront lues depuis le fichier qui est
                            indiqu� plut�t que depuis le terminal.\\
\hline

\verb'commande1 | commande2' & Les deux commandes sont ex�cut�es simultan�ment,
                               l'entr�e standard de la seconde �tant connect�e
                               par un tube (pipe) � la sortie standard de la 
                               premi�re.\\
\hline
\end{tabular}
\begin{center}
  \caption{Syntaxe de redirection de sortie standard vers une autre commande}
  \label{tab:redir_commandes}
\end{center}
\end{table}


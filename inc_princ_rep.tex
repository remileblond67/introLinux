\begin{table}[htbp]

\begin{center}
\begin{tabular}{|l|p{12cm}|}
\hline
R�pertoire & Description du contenu \\
\hline
/bin  &  Fichiers ex�cutables n�cessaires � l'initialisation\\
/boot &  Noyau et fichiers n�cessaires au d�marrage\\
/dev  &  Fichiers sp�ciaux d�crivant les p�riph�riques du syst�me\\
/etc  &  Fichiers de configuration du syst�me\\
/home &  R�pertoires personnels des utilisateurs\\
/lib  &  Librairies syst�me et modules\\
/lost+found &   Fichiers retrouv�s par fsck. On retrouve ce
                r�pertoire au point de montage de toutes les partitions
                mont�es avec le droit d'�criture.\\
/mnt  &  Points de montage des syst�mes de fichiers non permanents (CD-ROM,
         disquettes... )\\
/proc &  Syst�me de fichiers virtuel d�crivant le fonctionnement du syst�me
         (utilisation des ressources, �tat des processus... )\\
/root &  R�pertoire personnel de l'administrateur du syst�me (super
         utilisateur)\\
/sbin &  Fichiers ex�cutables r�serv�s � l'administration du syst�me\\
/tmp  &  Fichiers temporaires\\
/usr  &  Programmes, librairies et fichiers accessibles en lecture seule\\
/var  &  Donn�es variables li�es � la machine (spool, traces... )\\
\hline
\end{tabular}
\end{center}
  \caption{Pr�sentation rapide des principaux r�pertoires d'un syst�me de fichier Linux}
  \label{tab:princ_rep}
\end{table}

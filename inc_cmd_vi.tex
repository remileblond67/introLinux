\begin{table}
  \begin{center}
\begin{tabular}{|c|p{10cm}|}
\hline
Commande&Explication\\
\hline
a         & Ajoute du texte apr�s la position courante\\
A         & Ajoute du texte � la fin de la ligne courante\\
i         & Ajoute du texte avant la position courante\\
I         & Ajoute du texte au d�but de la ligne courante\\
o         & Ajoute une nouvelle ligne apr�s la ligne courante\\
O         & Ajoute une nouvelle ligne avant la ligne courante\\
x         & Supprime le caract�re courant\\
r         & Remplace le caract�re courant\\
R         & Remplace les caract�res suivants (mode "refrappe")\\
dd        & Efface la ligne courante\\
dw        & Efface le mot courant\\
d\$       & Efface la fin de la ligne courante\\
cw        & Change le mot courant\\
c\$       & Change la fin de la ligne courante\\
c\^       & Change le d�but de la ligne courante (de la position 
            courante au d�but de la ligne)\\
J         & Joint la ligne courante et la ligne suivante\\
<num�ro>G & Atteint un num�ro de ligne\\
G         & Atteint la fin du fichier\\
:q        & Quitte l'�diteur (provoque une erreur si le fichier a �t�
            modifi� depuis le dernier enregistrement).\\
:q!       & Quitte l'�diteur en ignorant les modifications apport�es 
            au fichier\\
:w        & Enregistre le fichier courant\\
:wq ou :x & Quitte l'�diteur en enregistrant les modifications du
            document\\
/<texte>  & Recherche le texte � partir de la position courante en
            descendant dans le fichier\\
?<texte>  & Recherche le texte � partir de la position courante en 
            remontant dans le fichier\\
n         & Atteint la prochaine occurrence du texte recherch� (dans le
            sens d�fini pour la recherche)\\
u         & Annule le r�sultat de la derni�re commande\\
.         & Recommence la derni�re commande\\
m<car>    & Marque la ligne courante. La marque porte le nom du caract�re suivant ``m''\\
'<car>    & Retourne � une ligne marqu�e.\\
\hline
\end{tabular}
\end{center}
\caption{Commandes de l'�diteur vi}
\label{tab:com_vi}
\end{table}

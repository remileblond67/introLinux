\begin{table}[htbp]
  \begin{center}
  \begin{tabular}{|c|l|p{8cm}|}
  \hline
  Commande & Syntaxe & Description\\
  \hline
\com{ps} & \com{ps} & Permet de consulter la liste des processus en cours. Par d�faut, seuls les processus lanc�s depuis l'interpr�teur de commande courant sont list�s. L'option '-edf' permet de retourner la liste exhaustive des processus actifs dans le syst�me. \\
\com{kill} & \com{kill} -signal pid & Permet de demander l'arr�t de l'ex�cution d'un processus. L'option '-9' est plus violente et force l'arr�t imm�diat du processus, sans passer par la proc�dure de sortie �ventuellement pr�vue dans le programme. Cette option est � utiliser lorsque le processus ne r�pond pas � la premi�re "sommation".\\
\com{jobs} & \com{jobs} & Permet de consulter la liste des travaux lanc�s en arri�re plan depuis l'interpr�teur de commande courant. \\
\com{top} & \com{top} & Application permettant de surveiller en temps r�el l'activit� du syst�me (processus actifs, utilisation m�moire et processeur... ). \\
  \hline
  \end{tabular}

  \caption{Commandes de gestion des processus}
  \label{tab:cmd_manip_proc}
  \end{center}
\end{table}

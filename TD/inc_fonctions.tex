% Paquetages � utiliser
\usepackage[T1]{fontenc}
\usepackage{babel, indentfirst}
\usepackage[xdvi]{graphicx}
\usepackage{latexsym}
\usepackage{tabularx}
\usepackage{makeidx}
\usepackage{times}
\usepackage{color}
\usepackage[ps2pdf]{hyperref}

% Mise en forme des noms de commande
\newcommand{\com}[1] {\texttt{#1}\index{#1 \textit{(cmd)}}}
% Mise en forme des exemples de commande
\newcommand{\comex}[1] {\texttt{#1}}
% Mise en forme des noms de fichier
\newcommand{\fich}[1] {\texttt{#1}\index{#1 \textit{(fich)}}}
% Mise en forme des noms de r�pertoire
\newcommand{\rep}[1] {\texttt{#1}\index{#1 \textit{(r�p)}}}
% Renvoi vers un label
\newcommand{\voir}[1] {voir \ref{#1}, page \pageref{#1}}
% Renvoi vers une figure
\newcommand{\voirf}[1] {voir figure \ref{#1}, page \pageref{#1}}
% Renvoi vers un tableau
\newcommand{\voirt}[1] {voir tableau \ref{#1}, page \pageref{#1}}
% Mise en forme du r�sum� de chaque partie
\newcommand{\resumchap}[1] {
  \textit{#1}
}
% Insertion d'une image EPS en pleine largeur de page
\newcommand{\imagelarg}[3] {
\begin{figure}[htbp]
  \begin{center}
    \includegraphics[width=\linewidth]{../eps/#1}
    \caption{#2}
    \label{#3}
  \end{center}
\end{figure}
}
% Insertion d'une image EPS
\newcommand{\image}[3] {
\begin{figure}[htbp]
  \begin{center}
    \includegraphics{../eps/#1}
    \caption{#2}
    \label{#3}
  \end{center}
\end{figure}
}
% Insertion d'une image EPS
\newcommand{\imagel}[4] {
\begin{figure}[htbp]
  \begin{center}
    \includegraphics[width=#1cm]{../eps/#2}
    \caption{#3}
    \label{#4}
  \end{center}
\end{figure}
}

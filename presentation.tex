\chapter*{Pr�sentation du cours}

\section*{Objectifs du cours}
Le cours "syst�mes informatiques en r�seau" vise � fournir aux auditeurs
les comp�tences suivantes : Pour chacun des deux syst�mes Windows et Linux, �tre
capable de les installer, les administrer, cr�er et g�rer un r�seau dans ces
environnements, installer des logiciels bureautiques.

Ce cours est divis� en deux parties �gales de 50 heures, consacr�es
respectivement � l'�tude du syst�me Windows et du syst�me Linux. Ce document
traite uniquement du syst�me Linux.

\section*{Admission et pr�-requis}
Cette formation s'adresse � toute personne en situation d'emploi ou de
recherche d'emploi.  Elle est ouverte aux b�n�ficiaires
d'emplois-jeunes ou aux aides �ducateurs pr�parant leur insertion
professionnelle future.  Aucun dipl�me en informatique n'est � priori
exig�.  Il est cependant souhaitable d'avoir d�j� utilis� un
ordinateur et d'avoir une certaine pratique de son syst�me
d'exploitation.  Des connaissances g�n�rales correspondant au niveau
Baccalaur�at sont exig�es.  Un entretien pr�alable � l'inscription
permettra, le cas �ch�ant, de s'assurer du niveau d'entr�e du
candidat.


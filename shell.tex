\chapter{Les scripts shell}


Commande ls (-a, -l, expressions g�n�rales - glob * [], redirection
vers un fichier) Fichiers et r�pertoires : m�me chose, type diff�rents

Commande cat

Suppression de fichier (rm)

Cr�ation de r�pertoire (mkdir)

Suppression de r�pertoire (rmdir, rm -R)

Copie de fichier (cp) D�placement de fichier (mv) Liaison de fichiers
(ln statique ou physique)

Editeur vi

Recherche de fichiers (find, locate - updatedb) : -type [df], -name,
-size


Recherche et manipulation internes aux fichiers (More, grep, sort,
wc... )

Utilisation de pipes pour communication inter-processus (exemple
cat|more)

Mise en pratique : recherche des commandes utilis�es (ls, cd... )

Principe du PATH

Droits sur les fichiers et r�pertoires

Essayer de cr�er un fichier ou un r�pertoire dans /etc ou /bin

Commande id : voir le groupe de l'utilisateur

Ls -l : Insister sur propri�taire et groupe

Expliquer la signification du champ rwxrw-r--

\begin{table}[htbp]
  \begin{center}
  \begin{tabular}{|c|l|p{6cm}|}
  \hline
  Commande & Syntaxe & Description\\
  \hline
  \com{cp} & \com{cp} sources destination & Copie les fichiers source vers la destination. Il peut y avoir plusieurs sources, mais il n'y a toujours qu'une seule destination. S'il y a plusieurs sources, la destination est n�cessairement un r�pertoire.\\
  \com{mv} & \com{mv} sources destination & D�place les fichiers source vers la destination. \\
  \com{rm} & \com{rm} fichier(s) & Supprime un ou plusieurs fichier(s). L'option '-R' permet de parcourir le contenu des �ventuels r�pertoires.\\
  \com{ln} & \com{ln} -s fichier nom\_lien & Cr�e un lien vers un fichier existant. L'option '-s', qui permet de cr�er un lien symbolique au lieu d'un lien physique, est a privil�gier car la gestion des liens physiques peut tr�s vite devenir complexe.\\
  \hline
  \end{tabular}
  \caption{Commandes de manipulation de fichiers}
  \label{tab:cmd_manip_fich}
  \end{center}
\end{table}
